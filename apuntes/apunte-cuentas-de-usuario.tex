%%% LaTeX Template: Article/Thesis/etc. with colored headings and special fonts
%%%
%%% Source: http://www.howtotex.com/

\documentclass[12pt]{article}


\usepackage{apuntes-estilo}
\usepackage{fancyhdr,lastpage}



\def\maketitle{

% Titulo 
 \makeatletter
 {\color{bl} \centering \huge \sc \textbf{
Administrando Cuentas de Usuario \\
% \large \vspace*{-8pt} \color{black} Vim el Editor de Seis Billones de Dólares
 \vspace*{8pt} }\par}
 \makeatother


% Autor
 \makeatletter
 {\centering \small 
 	Departamento de Ingeniería de Computadoras \\
 	Facultad de Informática - Universidad Nacional del Comahue \\
 	\vspace{20pt} }
 \makeatother

}

% Custom headers and footers
\fancyhf{} % clear all header and footer fields
\fancypagestyle{plain}{\fancyhf{}}
  	\pagestyle{fancy}
 	\lhead{\footnotesize  Administrando Cuentas de Usuario Linux - Departamento de Ingeniería de Computadoras}
 	\rhead{\footnotesize \thepage\ }	% "Page 1 of 2"

\def\ti#1#2{\texttt{#1} & #2 \\ }



\begin{document}

\thispagestyle{empty}
\maketitle
\setlength{\parindent}{0pt}




%\textit{Similitudes entre administradores de sistema y
%narcotraficantes: ambos miden cosas en Kilos y tienen usuarios} (Viejo y
%cansador chiste de computación)
\section*{Introducción}
La creación y borrado de usuarios es una tarea de rutina bastante 
común dentro de la administración de sistemas. Por este motivo, 
es necesario familiarizarse con varios aspectos relativos a las
cuentas de los usuarios, por lo que en este apunte se detallan 
las tareas de creación, modificación y borrado de usuarios dentro 
de un sistema GNU/Linux.

Del mismo modo las cuentas de usuario y su mantenimiento, son un 
factor clave en la administración de la seguridad del sistema.

Por otra parte, debido a la diversidad de distribuciones existentes,
las herramientas disponibles para la gestión de cuentas de usuario pueden 
variar de unas a otras (consulte la documentación
específica a su distribución GNU/Linux para obtener mayor información). Sin 
embargo, los cambios que estas herramientas implementan  en última instancia, 
son, en general, los mismos en todas las distribuciones.  

\subsection*{¿Qué es una cuenta?}
De manera general podemos decir que una cuenta es un nombre de usuario 
mas una contraseña, salvo excepciones, y todos los archivos 
(de configuración y aquellos personales) que
impliquen el ingreso y permanencia de un usuario en el sistema.

Cuando una computadora la utilizan muchos usuarios es necesario poder
diferenciarlos. Por ejemplo, para que sus archivos privados
permanezcan privados. Esto es importante aún si el sistema es usado por una
sola persona a la vez \footnote{Recuerde que los sistemas GNU/Linux son 
en esencia de tiempo compartido, esto es, múltiples usuarios pueden ejecutar
procesos concurrentemente en un dado momento, sin interferir unos con otros.},
como sucede con la mayoría de las computadoras personales.

Es así que, a cada usuario se le da un nombre de usuario alfanumérico  
\textbf{único}, y ese nombre es usado para ingresar al sistema. 
Además del nombre, el sistema le asigna un identificador numérico único o 
\texttt{uid} (del inglés user identification), que le permite manipular todo 
lo relativo al usuarios más eficientemente que si tuviese que manejar el 
nombre alfanumérico, que es más apropiado para los seres humanos.   
Luego una \textit{cuenta} es: todos los archivos, recursos, e información 
que pertenece a un usuario. El término insinúa algo similar a 
a lo que ocurre en bancos y en sistemas comerciales, en donde cada cuenta 
usualmente tiene algo de dinero asignado, y ese dinero se gasta a 
diferentes velocidades dependiendo de cuantos usuario exijan el sistema.
Dentro de un sistema operativo, una cuenta, por ejemplo  puede tener una 
cantidad de espacio de disco que puede tener un precio por mega por día, y
tiempo de procesamiento que puede tener un precio por segundo, por citar algunos
ejemplos. 

\section*{Crear una cuenta de usuario}
El núcleo Linux en sí trata a los usuario como meros números. Cada
usuario es identificado por un único número entero, el \textit{uid
(identificación de usuario)}. Esto es debido a que, para un sistema, un número es mas fácil
y rápido de procesar que un nombre. Una base de datos o tabla
asociada a dichos \texttt{uid}, por fuera del núcleo, asigna un nombre alfanumérico 
único a cada \texttt{uid}. Dicha base de datos también contiene información adicional, 
como su nombre completo, el intérprete de comandos predeterminado, etcétera. 

Para crear un usuario, se necesita agregar información sobre el mismo a la
base de datos y crear un directorio ``inicio'' (directorio principal
del usuario, en inglés ``home'') para él. También puede ser necesario educar al 
usuario, y configurar un ambiente conveniente para él (por ejemplo definiendo sus
variables de entorno de lenguaje, LANG, la hora local en la que se encuentra
TIMEZONE, etc).

La mayoría de las distribuciones de GNU/Linux cuentan con programas para crear
cuentas. Existen varios programas disponibles en los repositorios,
o en Internet \footnote{En Internet puede comenzar buscando en:
http://sourceforge.com y http://freshmeat.com}.  
Los comandos mas utilizados son \texttt{\textbf{adduser}} y
\texttt{\textbf{useradd}}. También existen herramientas gráficas, usualmente 
cada entorno de escritorio gráfico (como ser GNOME, KDE, etcétera) proveen  
implementaciones propias. 

Más allá de la herramienta particular que se utilice,  
el trabajo manual requerido para crear una nueva cuenta es sencillo. En la sección
\textit{Crear un usuario a mano} se describe todos los detalles relacionados. 
Las herramientas particulares están orientadas a mantener la consistencia entre 
las distintas tareas que involucra crear un usuario, pero pueden ser reemplazadas
en su totalidad por un grupo de tareas concretas que veremos más adelante. 


\subsection*{\texttt{/etc/passwd} y \texttt{/etc/shadow}}

La base de datos básica de usuarios en un sistema Unix, o GNU/Linux en particular, 
es un \textbf{archivo de texto} \texttt{/etc/passwd} (llamado el \textit{archivo de
contraseñas}\footnote{Este nombre se mantiene por compatibilidad y razones históricas, aún cuando 
en los sistemas modernos dicho archivo ya no contenga las contraseñas efectivamente.}), 
que contiene todos los nombres de usuarios válidos y su información asociada. 

Este archivo es consultado por el sistema en cada intento de ingreso al mismo (login).
El archivo tiene una línea por usuario, y es dividido en siete campos delimitados por 
dos puntos ``:'' en el siguiente orden:

	\begin{enumerate}
	\item{Nombre de usuario: alfanumérico y sensible a mayúsculas y minúsculas.}
	\item{Contraseña, de modo cifrado: es opcional. Usualmente en este campo encontraremos 
	una ``x'', lo que indica el uso del archivo \texttt{/etc/shadow} como base de datos de 
	contraseñas}
	\item{Número de identificación de usuario (uid).} 
	\item{Número de identificación de grupo (gid): se refiere al grupo primario del usuario. Puede 
	pertenecer a más grupos, lo cual se especifica en el archivo \texttt{/etc/group}}.
	\item{Nombre completo u otra información descriptiva de la cuenta.}
	\item{Directorio inicio (directorio principal del usuario, en inglés ``home'')}
	\item{Intérprete de comandos (programa a ejecutar al ingresar al sistema).}
	\end{enumerate}

El formato esta explicado con mas detalles en la página de manual correspondiente: 
\texttt{man 5 passwd} (sección 5 ``Formato de ficheros y convenios'').

\textit{Acerca del uso del segundo campo y el archivo \texttt{/etc/shadow}}: 

Cualquier usuario del sistema debe poder leer el archivo de \texttt{/etc/passwd}
ya que, entre otras cosas, muchos programas ejecutados por el usuario requieren de este acceso 
para obtener información y funcionar correctamente. Esto significa que: si almacenamos 
la contraseña en el segundo campo, esta también estará disponible para todos. En principio, 
esto no parece ser un problema porque dichas contraseñas se encuentran cifradas\footnote{La 
contraseña no es legible a primera vista por un ser humano, no se encuentra almacenada en 
texto plano}.  
Sin embargo, dicho cifrado puede ser quebrado, sobre todo si la contraseña es ``débil'' (por ejemplo, 
una palabra de diccionario).  \footnote{Estadísticamente, según el estudio de
los métodos para romper claves cifradas, se ha establecido que aumenta
significativamente la seguridad una suma de características: tener más de seis
caracteres, combinar letras mayúsculas y minúsculas, a la vez que intercalar
también números.}. 


\fcolorbox{black}{grey}{
\parbox[t]{1.0\linewidth}{ \vspace*{0.4cm}
{\bf Ejemplo:} Si observamos los permisos de la base de datos de usuarios, vemos que 
si bien el archivo pertenece al usuario root, todos los usuarios del sistema pueden 
leer su contenido (sólo root puede modificarlo). 

{\tt
-rw-r--r-- 1 root root 2153 nov  6 12:03 /etc/passwd
}
\vspace*{0.4cm} } }

De lo anterior podemos concluir que tener las contraseñas almacenadas en 
en una base de datos separada, 
no accesible por cualquier usuario, es una buena medida de seguridad. Es por este motivo 
que existe el archivo \texttt{/etc/shadow} en la gran mayoría, sino todos, los sistemas
GNU/Linux modernos.  

El archivo \texttt{/etc/shadow} es una alternativa en la manera de
almacenar las contraseñas: las claves cifradas se guardan en un archivo
en dicho archivo que solo puede ser leído y modificado por el administrador del 
sistema\footnote{Verifique los permisos en su instalación. En algunos casos se crea 
un grupo especial ``shadow'' al que se le permite leer dicho archivo.}. Así el 
archivo \texttt{/etc/passwd} solo contendrá un marcador especial en ese segundo campo, en 
lugar de la contraseña en sí misma. 
El formato de \texttt{/etc/shadow} está explicado con mas detalles en la página de 
manual correspondiente: \texttt{man 5 shadow} (sección 5 ``Formato de ficheros y convenios'').


\subsection*{Elegir números de identificación de usuario y grupo}
En general dentro de muchos sistemas, no es importante cuál es el número de 
identificación de usuario y grupo (uid y gid) más allá de los límites del sistema. 
Si bien suele reservarse el rango cero a cien 
para usuarios de sistema y aplicaciones (por ejemplo: sys y daemon, es decir usuarios 
que no se corresponden con personas reales), esto es sólo una convención, 
a excepción del uid cero que corresponde a root en cualquier caso.  

Sin embargo cuando, por ejemplo, se utilizan sistemas sistemas de archivos de red, como 
NFS (Network file System), el número de identificación deber traspasar
los límites del sistema. Es decir, se necesita que la asociación entre un nombre de 
usuario y los números de identificación de usuario (uid) y grupo (gid) sean los mismos en todos
los sistemas. Esto es porque el sistema de archivos de red también identifica al
usuarios (nombre de usuario) con su respectivo uid.  

Por lo anterior, si se desea usar un sistema de archivos de red NFS,
será necesario implementar un mecanismo 
para sincronizar la información de cada cuenta. Una alternativa para esto 
es el sistema NIS (del inglés ``Network Information Service'') o LDAP (del inglés Lightweight 
Directory Access Protocol). La descripción de estos servicios 
esta fuera del alcance de este curso. Sea cual sea la solución elegida, esta deberá 
garantizar la unicidad de nombres y uids entre los sistemas que comparten 
archivos a través de NFS u otros servicios del estilo, caso contrario se expone
el sistema, y en particular el contenido de los sistemas de archivos NFS, a 
problemas de seguridad y confusión de los usuarios. 

\fcolorbox{black}{grey}{
\parbox[t]{1.0\linewidth}{ \vspace*{0.4cm}
{\bf Ejemplo:} Supongamos que en la computadora A, el uid 300 corresponde al usuario jose, 
y que en la computadora B, dicho uid corresponde al usuario andrea. Si la computadora A
exporta el sistema de archivos \texttt{/export/compartidosA} por NFS a la computadora B. Sucederá
que, cuando andrea inicie sesión en B, y revise el contenido del directorio \texttt{/export/compartidosA}
(puede que en B, el directorio se encuentre montado en otro lugar dentro de la jerarquía de 
directorios, por simplicidad mantenemos el mismo nombre de origen), todos los archivos 
pertenecientes a jose en A, pertenecerán a andrea en B. Esto es un grave problema 
de seguridad.
\vspace*{0.4cm} } }




\subsection*{Crear un usuario manualmente}

Para crear una nueva cuenta manualmente, se necesita seguir los siguientes pasos:

\begin{itemize}
	
	\item  Editar \texttt{/etc/passwd} con el comando 
	\texttt{\textbf{vipw}} y agregar una nueva línea por cada nueva cuenta.
	Teniendo cuidado con la sintaxis (respetar los siete campos en el orden 
	descripto anteriormente).  \textit{\bf ¡No lo edite directamente
	con cualquier editor!}. Utilice el comando \texttt{\textbf{vipw}}\footnote{
	A menos que las variables de entorno VISUAL o EDITOR contengan algo diferente o 
	no se encuentren definidas, vipw utilizará el editor vi para la edición}, 
	el cual bloquea el archivo, así otros comandos no podrán actualizarlo al mismo tiempo. 

	\item En caso de utilizar el archivo \texttt{/etc/shadow}, para almacenar
	la contraseña cifrada, edite el mismo con el comando \texttt{vipw -s}. En este
	caso la línea correspondiente en \texttt{/etc/passwd}, deberá contener el caracter
	``x'' en el segundo campo. Agregue una nueva línea por cada nueva cuenta. Inicialmente
	el campo correspondiente a la contraseña deberá contener el caracter ``!''. Luego 
	se actualizará con el cifrado correspondiente, al utilizar el comando \texttt{passwd} para 
	configurar una contraseña inicial.  

	\item Similarmente, edite \texttt{/etc/group} con
	\texttt{\textbf{vigr}}, si necesita crear también un
	grupo especial para el usuario, o bien agregarlo a otros grupos
	diferentes del grupo primario especificado en el archivo \texttt{/etc/passwd}.  
        
	\item Cree el directorio de inicio (``home'') del
	usuario con el comando \texttt{\textbf{mkdir}}. Usualmente el nombre del directorio
	de inicio del usuario coincidirá con el nombre alfanumérico del mismo, sin embargo 
	esto es sólo una convención. 

	\item Copie los archivos de \texttt{/etc/skel} al nuevo directorio creado. 
	En la sección a continuación se detalla el contenido de /etc/skel. 

	\item Modifique la pertenencia del dueño y permisos con los comandos
	\texttt{\textbf{chown}} y \texttt{\textbf{chmod}} (Ver páginas de
	manual de los respectivos comandos y apunte de cátedra de permisos). 
	La opción \texttt{-R} es muy útil para aplicar los cambios sobre todos 
	los archivos del directorio y subdirectorios en un sólo paso (recursivo).
	Recuerde hacer esta operación con cuidado y utilizando los argumentos 
	correctos, \texttt{chown -R user *} ejecutado como root en el lugar equivocado,
	digamos /, puede causar una catástrofe (similar para chmod). En lugar de 
 	sólo * utilice rutas, por ejemplo \texttt{chown -R user /home/user/}. 
 
	\item Por último asigne una contraseña con el comando
	\texttt{\textbf{passwd}}. La opción ``\texttt{-e}'' exige al nuevo usuario 
	que cambie su contraseña inicial en su primer inicio de sesión, lo cual 
	es equivalente a escribir un cero (0) en el tercer campo del archivo 
	/etc/shadow. Esto es recomendable para evitar que los usuarios utilicen 
	sus passwords iniciales configurada por el administrador.  

\end{itemize} 

\fcolorbox{black}{grey}{
\parbox[t]{1.0\linewidth}{ \vspace*{0.4cm}
{\bf Ejemplo:}  \\
{\tt
\# grep jgonzalez /etc/passwd 

jgonzalez:x:1002:1003:Javier Gonazalez:/home/jgonzalez:/bin/bash

\# grep jgonzalez /etc/shadow 

jgonzalez:\$6\$s3PaFwKz\$JcaBEhIuOapwiypiRUm1iIhzYD/qyXwqh.f4b6m \\ 
IBexyhUbqV2ojnh71EBSThbxV05fd67MpT1z/f1IGyhTat1:0:0:99999:7::: 

\# grep 1003 /etc/group 

security:x:1003:

\# grep jgonzalez /etc/group 

cdrom:x:1000:perez,galindo,jgonzalez

\# mkdir /home/jgonzalez

\# cd /etc/skel ; cp -r . /home/jgonzalez/

\# chown -R jgonzalez.security /home/jgonzalez 

\# chmod -R go=u,go-w /home/jgonzalez 
}
\vspace*{0.4cm} } }

La secuencia anterior realiza lo siguientes pasos:
\begin{enumerate}
	\item Verificar la entrada que creamos en /etc/passwd y /etc/shadow. 
	\item Verificar la existencia del grupo primario.  En este caso jgonzalez tiene 
	      asignado como grupo primario el grupo security (gid 1003). 
	\item Verificar grupos secundarios. En este caso jgonzalez pertenece al grupo cdrom (gid 1000).  
	\item Crear el directorio de inicio (home). 
	\item Copiar archivos predeterminados desde /etc/skel (opcional). 
	\item Modificar la pertenencia del nuevo directorio y su contenido, tal que este pertenezca al 
	jgonzalez (dueño) y al grupo security. 
	\item Asignar permisos al grupo security y al resto tal que: sean 
	iguales los permisos asignados al propietario jgonzalez (\texttt{go=u}),
        a excepción del permiso 
	de escritura que se remueve (\texttt{go-w}). Si quisiéramos negar
	todos los permisos al grupo y resto de los usuarios 
	podríamos hacerlo usando ``\texttt{chmod go= .}'' en lugar de la 
	sentencia más arriba. 
\end{enumerate}

Como se observa, la creación manual de usuarios consiste de una serie de 
pasos sencillos. Sin embargo, el olvido de uno de estos pasos puede dar 
lugar a inconsistencias y fallas de seguridad. Es por ello que se sugiere
el uso de herramientas como adduser/useradd, usermod, passwd para 
realizar actualizaciones y creación de usuarios consistente. 

{\bf Usuarios especiales:} 

En ciertas ocasiones es necesario crear cuentas que no corresponden 
a personas físicas. Muchas de ellas, de hecho son creadas durante la 
instalación del sistema, o bien de una nueva aplicación. Por ejemplo esto 
sucede al instalar los motores de base de datos postgres o mysql. También 
podría ser útil crear este tipo de cuentas manualmente, por ejemplo para 
proveer un servicio de transferencia de archivos FTP (File Transfer Protocol) 
anónimo. Las cuestiones de seguridad relativas a esta implementación están fuera
del alcance de este curso. Algo importante acerca de este tipo de cuentas, es
que normalmente no se les permite el ingreso al sistema a través del programa 
login (por ejemplo en tty locales o a través del servicio de ssh), ya que se 
supone son utilizadas por aplicaciones internamente y no por usuarios en forma 
directa. Es por ello que estas cuentas suelen tener el caracter ``!'' o ``*'' en 
el segundo campo del archivo  \texttt{/etc/shadow} (correspondiente a la contraseña 
cifrada para usuarios físicos). Una política de seguridad adicional, en los 
casos que corresponda, consiste en no asignar un shell predeterminado, y en su lugar
utilizar \texttt{/bin/false} en el campo correspondiente en \texttt{/etc/passwd}.  

\subsection*{Ambiente inicial: \texttt{/etc/skel}}
Cuando se crea un directorio Inicio para un nuevo usuario, este es
inicializado por medio del directorio \texttt{/etc/skel}. 
\footnote{Apócope de la palabra inglesa ``skeleton'', que en castellano 
significa esqueleto, asiendo referencia al función de estructura.} El
administrador del sistema puede crear archivos dentro de
\texttt{/etc/skel} que proveerán un entorno predeterminado
para los usuarios. Por ejemplo, podría crear un archivo 
\texttt{/etc/skel/.profile} que configura las variable de entorno
de algún editor mas amigable para los usuarios nuevos, ó el lenguaje 
de preferencia, etc. 

En general es una buena política conservar dicho directorio lo mas pequeño
posible, ya que, en el futuro será imposible actualizar los archivos de
los usuarios. Por ejemplo, si cambia el nombre de un editor de texto a uno mas amigable,
todos los usuarios tendrán que editar su archivo  \texttt{.profile}.
El administrador del sistema podría tratar de hacer esto automáticamente con un
script \footnote{Lenguaje de programación cuyo código no necesita ser
compilado para ser ejecutado, por lo general es interpretado por el shell.}, pero
puede que las modificaciones personales de los usuarios realizadas a estos 
archivos, conflictúen con el script, resultando quizá en una corrupción de 
los archivos del usuario. Siempre que sea posible, es mejor poner aquello que 
corresponda a configuración global (es decir válida para todos los usuarios del sistema)
dentro de archivos globales, como por ejemplo /etc/profile. De esta manera es posible
actualizarlo sin corromper la configuración de ningún usuario.  


\section*{Cambiar las propiedades del usuario}
Con el correr del tiempo puede que se necesiten actualizar algunas de las 
propiedades de las cuentas. Por ejemplo, el usuario puede desear un shell 
predeterminado distinto al que posee actualmente. Los siguientes comandos 
permiten cambiar dichas propiedades: 

\begin{itemize}
\item 	\texttt{chfn}: cambia el campo del nombre completo.
\item 	\texttt{chsh}:  cambia el campo del interprete de comandos (shell).
\item 	\texttt{passwd}: cambia la contraseña. 
\item 	\texttt{usermod}: Los tres comandos anteriores pueden ser utilizados 
por cualquier usuario y afectarán a su propia cuenta. El comando \texttt{usermod}
está pensado para ser ejecutado por el administrador del sistema (root), y 
permite modificar prácticamente todas las propiedades de la cuenta de cualquier 
usuario. Véase la página del manual correspondiente para más información
(\texttt{man usermod})
\end{itemize}


Otras tareas pueden ser necesarias hacerlas manualmente. Por ejemplo,
para agregar o quitar a uno o varios usuarios
de uno o más grupos, editando \texttt{/etc/group} (con
\texttt{\textbf{vigr}}). Este tipo de tareas tienden a ser mas raras, de todas
maneras, siempre hay que ir con cuidado: si cambia un nombre de usuario, dicho
usuario dejara de acceder a su cuenta de correo a menos que también le genere un
alias a su dirección de correo.
	
\section*{Borrando usuarios}

Para borrar un usuario, primero borre los archivos que le pertenezcan:
casilla de correo, alias de correo, trabajos de impresión, trabajos pendientes a
través de los demonios \texttt{\textbf{cron}} y \texttt{\textbf{at}}, y
cualquier otra referencia al usuario.  Entonces quite las 
líneas relevantes de los archivos \texttt{/etc/passwd},  \texttt{/etc/shadow} y
\texttt{/etc/group} (recuerde borrar al usuario de todos los grupos
a los cuales pertenecía).  Puede ser buena idea deshabilitar la cuenta antes de
empezar a borrar cosas para prevenir que el usuario use la cuenta mientras esta
siendo eliminado.

Recuerde que los usuarios pueden tener archivos fuera de su directorio
Inicio.  Para encontrarlos use el comando:

 \texttt{find / -user <nombre-de-usuario>} 

El comando find puede tomar \textit{\bf mucho tiempo} si
tiene discos muy grandes o si monta un disco de red.

Algunas distribuciones tiene comandos especiales para realizar esta tarea. Por 
ejemplo \texttt{\textbf{deluser}} o \texttt{\textbf{userdel}}. Sin embargo, puede 
que dichos comandos no eliminen todo lo relativo al usuario, por lo que una 
revisión manual en muchos casos es conveniente.  



\section*{ Deshabilitar un usuario temporalmente}

A veces es necesario deshabilitar una cuenta temporalmente, sin borrarla.
Por ejemplo, un usuario pudo dejar de pagar el servicio, o el administrador de
sistema puede sospechar que un cracker	\footnote{En este caso traducible a 
invitado inesperado y malicioso} tiene la contraseña de esa cuenta.

Existen varias formas de lograr esto, la más común tal vez es agregar un caracter 
``!'' delante de la contraseña en el archivo \texttt{/etc/shadow}, de hecho esto 
es lo que hace el comando \texttt{usermod -L <nombre-de-usuario>} (la opción -U
de dicho comando realiza la operación inversa). Otra manera de deshabilitar una 
cuenta es cambiar su intérprete de comandos por otro programa, por ejemplo por 
\texttt{/bin/false} u otro  que solo envíe mensajes a la pantalla, por 
ejemplo uno que indique contactar con el administrador del sistema. De esta manera, 
cualquiera sea la forma que intente entrar al sistema con esa cuenta fracasará, 
y sabrá porqué. 

Una manera simple de crear un programa especial para reemplazar el shell, 
es escribir un ``script'' como el siguiente: 

Crear un archivo \texttt{/usr/local/lib/no-login/security}, y darle permisos de 
ejecución \texttt{chmod 755 /usr/local/lib/no-login/security}. Editar
el archivo y agregar el siguiente contenido: 

\begin{verbatim} 
#!/bin/bash 
echo Esta cuenta ha sido cerrada por razones de seguridad.
echo Por favor llame al 0800-333-afuera
\end{verbatim}

Los primeros dos caracteres ('\texttt{\#!}') le dicen al núcleo que el
resto de la línea es un comando que necesita ejecutarse por medio de un
intérprete. Luego de ejecutarse las dos líneas a continuación, 
se cerrará la conexión al usuario. 

\colorbox{grey}{\parbox[t]{0.95\linewidth}{ \vspace*{0.5cm} 
{
Por ejemplo si el usuario billg es sospechado
de infringir la seguridad, el administrador del sistema 
puede hacer algo como esto: 

\tt  
\# chsh -s /usr/local/lib/no-login/security billg \\
\# su - billg \\
 \\
    Esta cuenta ha sido cerrada por razones de seguridad. \\
    Por favor llame a 0800-333-afuera \\
\#  
} \vspace*{0.5cm} } } 

El propósito del comando \textbf{su} es verificar que el cambio
funciona, por supuesto.  

Es recomendable que este tipo de scripts se mantengan en un directorio
separado, así sus nombres
no interfieren con los comandos de los usuarios normales.


\include{licGASL}

\end{document}
